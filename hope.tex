\documentclass[]{article}

%opening
\title{Is there hope for the 11 year old? \\Or is Global Warming like the Black Death?}
\author{Simon Crase}

\begin{document}

\maketitle
The Black Death killed 100 million people. "But at length it came to Gloucester, yea even to Oxford and to London, and finally it spread over all England and so wasted the people that scarce the tenth person of any sort was left alive."--Geoffrey the Baker, \textit{Chronicon Angliae}. Global Warming is different: unlike the people of the 14th century, we know the cause, we know what is needed to fix it, and we have some ideas of how to go about it.


Put simply, we need to burn a lot less fossil fuels, but "the Devil is in the details". We have a number of solution that can be used today: solar, wind, and nuclear power; electric vehicles can be a good way to store excess solar and wind energy; telecommuting, where people stay at home to work or study and travel less; better public transport, to move people with less fuel. Even with the best will in the world, we won't eliminate all fossil fuels soon, so maybe we should try to mimic some of the processes that Nature uses to lock carbon up in rocks. Mother Nature tries to do a good job, but she's too slow.

Will all that be enough? It doesn't seem as if we have the all technology today to solve the whole problem, but that is no excuse for not making a start. If we want a world that we would like to live in we need to do more, and you can help in many ways: you might become a scientist or engineer developing new power sources, or getting rid of $CO_{2}$, or just making things more efficient; you can explore opportunities to use less fuel in your own life: use a bike, telecommute, avoid fast food (burger flippers can't telecommute); help convince people that there is a problem, and that we can solve it if we want.

"Optimism is a strategy for making a better future. Because unless you believe that the future can be better, you are unlikely to step up and take responsibility for making it so. If you assume that there's no hope, you guarantee that there will be no hope. If you assume that there is an instinct for freedom, there are opportunities to change things, there's a chance you may contribute to making a better world. The choice is yours."---Noam Chomsky
\end{document}
