\documentclass[]{article}

%opening
\title{Is it Warming? Is It Us? How Do We Know?}
\author{Simon Crase}

\begin{document}

\maketitle


Climate change deniers like to say "correlation isn't causation", which means that we can't necessarily conclude that one thing causes another just because they change at roughly the same time: the rooster crowing just before dawn doesn't \textit{make} the Sun rise. So it isn't enough to show that temperatures are rising. We must show that there is some way that humans can cause the temperature to rise, and show that this actually happens.
\paragraph{}
We have temperature measurements going back a couple of hundred years for land, from more and more measuring stations, and we have sea  and satellite measurements that are more recent. Some measurements go up, and others go down form one year to the next, which is what you might expect: when you go in a trip by car, sometimes you will turn to the left, sometimes to the right, and you may even appear to double back. Over a longer period of time (a few hours on a long journey), you will get nearer to your destination. And similarly, if you look at the average temperature over the earth, it will usually go up from one year to the next. If it goes down a little for a while, you will see it going up over a longer time. The "trend" of the temperature curve is upward.
\paragraph{}
How could we affect the temperature over the whole earth? A Swedish scientist, Svante Arrhenius, gave us an important clue in 1896, when he found that carbon dioxide ($CO_{2}$), in the Earth's atmosphere, can act like a blanket, and stop heat from escaping. We know that $CO_{2}$ is increasing all over the planet, because it can be measured in many different places. Like the temperature, the trend of $CO_{2}$ is upwards. \textit{But is it us?}
\paragraph{}
For the past 250 years or so we have been clearing forests, and burning coal and oil. Burning produces $CO_{2}$; since there are fewer trees to take up $CO_{2}$, the amount in the atmosphere increases. Scientists have added up the amount of $CO_{2}$ from coal and oil have found something strange: the increase in $CO_{2}$ is actually less than they expected. They are still trying to find out where the extra $CO_{2}$ has gone, but they are in no doubt that humans have caused the increase, and that this has caused the rise in temperature.
\end{document}
